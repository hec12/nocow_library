\documentclass[fleqn]{jarticle}

\usepackage{geometry} % See geometry.pdf to learn the layout options. There are lots.
\geometry{a4paper, top=2.0cm, left=2.0cm, right=2.0cm, bottom=2.0cm, includehead, includefoot}                   % ... or a4paper or a5paper or ... 

\usepackage{algpseudocode,algorithm}
\usepackage{amsmath}
\usepackage{url}

\renewcommand{\algorithmicrequire}{\textbf{Input:}}
\renewcommand{\algorithmicensure}{\textbf{Output:}}

\DeclareMathOperator*{\argmin}{arg\,min}
\DeclareMathOperator*{\argmax}{arg\,max}

\usepackage{color}
\usepackage{listings,jlisting}
\usepackage{courier}

\lstset{language=C++}
\lstset{columns=fullflexible}
\lstset{showspaces=false}

\lstset{numbers=left}
\lstset{numbersep=1zw}

\lstset{basicstyle=\small\ttfamily\footnotesize}

\lstset{keywordstyle={\bfseries \color[rgb]{0,0.5,0.85}}}
\lstset{commentstyle={\color[rgb]{0.8,0.0,0}}}
\lstset{morecomment=[l]{//}}
\lstset{stringstyle={\bfseries f\color[rgb]{0.0,0.6,0.0}}}

\lstset{frame={tb}}
\lstset{xrightmargin=0zw}
\lstset{xleftmargin=3zw}

\begin{document}

\title{Monge DP}
\author{ヘクト}
\maketitle

\section{はじめに}
Monge DPに必要な事項を忘れないようにするためのメモである.

\section{Monge性}
$
二変数の関数f(i,j)が「Monge」であるとは,任意の i \le j と k \le l について
$
\begin{align}
f(i,l) + f(j,k) \ge f(i,k) + f(j,l)
\end{align}
が成立することをいいます.Monge性はmin-max束の劣モジュラ不等式だとも言える.
\begin{align}
f(i,l) + f(j,k) \ge f(\min(i,j),\min(k,l)) + f(\max(i,j),\max(k,l)) = f(i,k) + f(j,l)
\end{align}
DPのコスト関数がMonge性を満たす時には,DPテーブルもMonge性を満たすことが知られている.
このMonge性を利用することにより,DPを高速化できることが知られている.

\section{Divide and Conquer Optimization}
この問題は,式(\ref{divide_dp})の形の漸化式で表されて,dp[n][k]を求めたい.
\begin{align}
&{\rm dp}[i][j] = \min_{0\le k<j}{{\rm dp}[i-1][k]+{\rm cost}[k][j]} \label{divide_dp} 
\end{align}
この式は普通のDPであるため,計算量 $O(kn^2)$ で計算できる.ここでMonge性を利用する.
式(\ref{divide_cut})を満たすC[i][j]をdp[i][j]が最小値を取るインデックスと定義する. 
\begin{align}
&{\rm C}[i][j] =\min_{0\le k<j}{{\rm dp}[i-1][k]+{\rm cost}[k][j]} \label{divide_cut}
\end{align}
Monge性より,C[i][j]は式(\ref{divide_cut_eq})の関係を満たす.
\begin{align}
&{\rm C}[i][j] \le {\rm C}[i][j+1]  \label{divide_cut_eq}
\end{align}

式(\ref{divide_cut_eq})の関係から,C[i][j]を分割統治法を用いることで計算量$O(kn\log{n})$で計算できる.  また,行最大値発見問題のアルゴリズムを用いることで
計算量$O(kn)$で計算できる.

\begin{lstlisting}[caption=Divide and Conquer,label=divide_code]
int dp[kmax][nmax];
void solve(int i,int L,int R, int optL,int optR){
	if(L>R) return;
	int M=(L+R)/2;
	int optM=-1;
	for(int k=optL;k<=optR;++k)
		if(dp[i+1][M]>dp[i][k]+cost[k][M])
			dp[i+1][M]=dp[i][k]+cost[k][M]	,optM=k;
	solve(i,L,M,optL,optM);
	solve(i,M,R,optM,optR);
	return;
}

int main(void){
	for(int i=0;i<=k;++i)
		for(int j=0;j<=n;++j)
			dp[i][j]=inf;
	dp[0][0]=0;
	for(int i=0;i<k;++i) solve(i,0,n,0,n);
	cout << dp[k][n] << endl;
	return 0;
}
\end{lstlisting}

\section{Knuth Optimization}
最適二分探索木問題を解くために,Knuthが開発した手法らしい.この問題は,式(\ref{knuth_dp})の形の漸化式で表される.
\begin{align}
&{\rm dp}[i][j] = {\rm cost}[i][j] + \min_{i<k<j}{{\rm dp}[i][k]+{\rm dp}[k][j]} \label{knuth_dp}
\end{align}
このDPは区間DPであるため,計算量$O(n^3)$で計算できる. ここでMonge性を利用する. 
式(\ref{knuth_cut})を満たすC[i][j]をdp[i][j]が最小値を取るインデックスと定義する. 
\begin{align}
&{\rm C}[i][j] =\argmin_{i<k<j}{{\rm dp}[i][k]+{\rm dp}[k][j]} \label{knuth_cut}
\end{align}
Monge性より,C[i][j]は式(\ref{knuth_cut_eq})の関係を満たす.
\begin{align}
&{\rm C}[i][j-1] \le {\rm C}[i][j] \le {\rm C}[i+1][j]  \label{knuth_cut_eq}
\end{align}

式(\ref{knuth_cut_eq})の関係から,幅が小さい方向からDPテーブルを埋めることにより,計算量$O(n^2)$で計算できる.  \\

\begin{lstlisting}[caption=Divide and Conquer,label=divide_code]
int dp[nmax][nmax];
int C[nmax][nmax];

int main(void){
	for(int i=0;i<=k;++i)
		for(int j=0;j<=n;++j)
			dp[i][j]=inf;
	for(int i=0;i<n;++i) dp[i][i]=0,C[i][i]=i;
	
	for(int d=2;d<=n;++d){
		for(int i=0;i+d-1<n;++i){
			int L=i,R=i+d-1;
			int idx=C[L][R-1];
			for(int j=C[L][R-1];j<=C[L+1][R];++j){
				if(dp[L][R]>dp[L][j]+dp[j+1][R]+cost[L][R])
					dp[L][R]=dp[L][j]+dp[j+1][R]+cost[L][R],idx=j;	
			}
			C[L][R]=idx;
		}
	}
	cout << dp[0][n-1] << endl;
	return 0;
}
\end{lstlisting}


\section{参考問題}
KUPC 2012 J Sashimi \\
JAG Summer 2013 Day2 D \\
JAG Autumn 2015 K \\


\begin{thebibliography}{9}
\bibitem{English} Letterfrequency(English),ALGORITMY.NET \\
\url{http://en.algoritmy.net/article/40379/Letter-frequency-English}
\bibitem{Hist} デジタル情報の処理と認識('12) 15章・鈴木一史 \\
\url{http://compsci.world.coocan.jp/OUJ/2012PR/pr_15_a.pdf}
\bibitem{Alice} Alice's Adventures in Wonderland(1866),Lewis Carroll \\
\url{https://en.wikisource.org/wiki/Alice%27s_Adventures_in_Wonderland_(1866)/Chapter_1}
\end{thebibliography}

\end{document}