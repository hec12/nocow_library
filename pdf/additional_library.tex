%\documentclass[9pt,a4paper,fleqn]{jarticle}
\documentclass[9pt,twocolumn,a4paper,landscape]{extarticle}
\usepackage{geometry} % See geometry.pdf to learn the layout options. There are lots.
\geometry{a4paper, top=0cm, left=1cm, right=1cm, bottom=1.0cm, includehead, includefoot}                   % ... or a4paper or a5paper or ... 

\usepackage[dvipdfmx]{graphicx}
\usepackage{bm}
\usepackage{comment}
\usepackage{amsmath, amsfonts}
\usepackage{subfigure, subfloat}
\usepackage{cite}
\usepackage{fancyvrb}
\usepackage{multirow}
\usepackage{comment}
\usepackage{subfigure}
\usepackage{url}
\usepackage[cache=false]{minted}

\DeclareMathOperator*{\argmin}{arg\,min}
\DeclareMathOperator*{\argmax}{arg\,max}

\newmintedfile[includecpp]{cpp}{
    linenos,
    mathescape,
    numbersep=5pt,
    frame=lines,
    framesep=2mm,
    tabsize=2
}

\title{Nocow Library (additional2)}
\author{ヘクト}
\date{\today}

\begin{document}
\maketitle

\begin{comment}
\begin{center}
\begin{minipage}{4cm}
	\begin{verbatim}
	  AA
	 ⊂・・⊃▼⌒ヽ
	 (ω_) )  ●|〜*
	   UU〜ーU
	\end{verbatim}
\end{minipage}
\end{center}
\end{comment}

%\tableofcontents


\section{Aho-Corasick}
\includecpp{../src/string/AHO_CORASICK.cpp}

\section{Heavy-Light-Decomposition}
\includecpp{../src/graph/HLD.cpp}

\section{線形連立方程式のメモ}

$\mathbf{Ax} = \mathbf{b}$ を $\mathbf{x}$ について解くとき,$\mathbf{A} \in \mathbb{R}^{n \times m},\mathbf{b} \in \mathbb{R}^n$


\begin{itemize}
	\item 決定系 $\text{rank}(\mathbf{A}) = n = m$ 解が唯一
	\item 優決定系 $\text{rank}(\mathbf{A}) = m < n$ 解がなし
	\item 劣決定系 $\text{rank}(\mathbf{A}) = n < m$ 解が複数
	\item ランク落ち $\text{rank}(\mathbf{A}) < \min(n,m)$ 線形従属になっている方程式を取り除くと、上の3つに収束
\end{itemize}

\end{document}