\section{幾何}

\subsection{注意事項}
\begin{itemize}
\setlength{\parskip}{0cm} % 段落間
\setlength{\itemsep}{0cm} % 項目間
\item sgn関数の定義はちゃんと写す・istream を使う時は pの代入を忘れない
\item 交点を求める前に交差判定が必要 iss(a,b) ill(a,b) == parallel(a,b)
\item angleの定義には気をつける
\item complexの比較関数はnamespace stdの中で書く
\end{itemize}

\subsection{ベクトル}
\includecpp{../src/geometry/VECTOR.cpp}

\subsection{点集合}

\subsubsection{凸包}
\includecpp{../src/geometry/CONVEX_HULL.cpp}

\subsubsection{最近点対}
\includecpp{../src/geometry/CPP.cpp}

\subsubsection{最遠点対}
\includecpp{../src/geometry/FPP.cpp}

\subsection{直線と線分}
\includecpp{../src/geometry/LINE.cpp}

\subsection{多角形}
\includecpp{../src/geometry/POLYGON.cpp}

\subsection{円}
\includecpp{../src/geometry/CIRCLE.cpp}

%\subsection{最小包含円}

\subsection{線分アレンジメント}
\includecpp{../src/geometry/SEGMENT_ARRANGEMENT.cpp}

\subsection{円アレンジメント}
\includecpp{../src/geometry/CIRCLE_ARRANGEMENT.cpp}

\subsection{ボロノイ図}
\includecpp{../src/geometry/VORONOI.cpp}
