\section{数学}

\subsection{素数判定}
\includecpp{../src/number_theory/prime.cpp}

\begin{comment}
\subsection{GCD/LCM} 
\includecpp{../src/number_theory/GCDLCM.cpp}
\end{comment}

\subsection{EXTGCD 中国剰余定理} 
\includecpp{../src/number_theory/chinese_remainder_easy.cpp}

\subsection{中国剰余定理} 
\includecpp{../src/number_theory/CHINESE_REMAINDER.cpp}


\subsection{オイラーの$\phi$関数}
\begin{equation}
\phi(n) = n \prod_{k=1}^{d}{\dfrac{p_{k}-1}{p_k}}
\end{equation}

\subsection{メビウスの反転公式}
\includecpp{../src/number_theory/MOBIUS_FUNCTION.cpp}


\subsection{高速ゼータ・メビウス変換}
\includecpp{../src/number_theory/FZT_FMT.cpp}

\subsection{行列} 
\includecpp{../src/number_theory/MATRIX.cpp}
\subsubsection{連立一次方程式} 
\includecpp{../src/number_theory/LINEAR_EQUATION.cpp}
%\subsubsection{行列式}
%\includecpp{../src/number_theory/DETERMINANT.cpp}

\subsection{ハンガリアン法}
\includecpp{../src/number_theory/HANGARIAN.cpp}

\subsection{基底変換}

\subsubsection{高速フーリエ変換}
\includecpp{../src/number_theory/FFT.cpp}

\subsubsection{高速剰余変換} 
\includecpp{../src/number_theory/NTT.cpp}

\subsubsection{高速アダマール変換} 
\includecpp{../src/number_theory/FHT.cpp}

\subsection{ラグランジュ補間} 
\includecpp{../src/number_theory/LAGRANGE_INTERPOLATION.cpp}

\subsection{公式集}

\begin{itemize}
\setlength{\parskip}{0cm} % 段落間
\setlength{\itemsep}{0cm} % 項目間
\item フェルマーの小定理: 素数$p$,任意の整数$x$に対し,$x^p \equiv x \pmod{p}$
\item 中国剰余定理: $k$個の整数$m_i$がどの2つも互いに素ならば,任意に与えられる$k$個の整数$a_i$に対し、$x \equiv a_i \pmod{m_i}$である$x$が一意に定まる.
\item ポリアの数え上げ定理: すべてのパターンをちょうど同じ回数だけ数え上げ,重複回数で割ることで数え上げが可能
\item シンプソン公式: 数値積分の公式 $\frac{b-a}{6}\left[ f(a)+4f(\frac{a+b}{2})+f(b) \right]$ 本来は近似値だが,$f(x)$が二次以下であれば厳密値が得られる. 
\end{itemize}